\section{RDF und Linked Data}
\label{sec:linked-data}

Einer der Vorreiter des World Wide Webs Tim Berners-Lee hat zusammen mit anderen Autoren \citeyear{berners2001semantic} ein Artikel im Scientific American Journal mit dem Titel, \citetitle{berners2001semantic} publiziert. In diesem Artikel haben die Autoren der Konzept des Semantic Webs eingeführt und eine Vision der Zukunft formuliert: \hyphenquote{german}{The Semantic Web will bring structure to the meaningful content of Web pages, creating an environment where software agents roaming from page to page can readily carry out sophisticated tasks for users.} Um diese Vision gerecht zu werden ist nach \citeauthor{blumauer2006semantic} der Begriff \hyphenquote{german}{Semantic Web} genauer als \hyphenquote{german}{Semiotic Web} zu verstehen. Aus Sicht der Semiotik setzt dieser Art von Interoperabilität zwischen Akteure im Semantic Web voraus, dass sie sich auf die syntaktischen, semantischen und pragmatischen Ebene verständigen können. Das heißt, wenn ein Sender eine Nachricht zum Empfänger Schickt, ist der Empfänger in der Lage die Nachricht richtig zu lesen (Syntax), zu interpretieren (Semantik), und schließlich richtig darauf zu reagieren (Pragmatik)\parencite[vgl.]{voigtmann2002enterprise}. Diese Art von Integration im Web wird in <cite> für kann zum Beispiel zwischen Unternehmen durch EbXML und Web-EDI geschehen (mit RDF?).  RDF unterstützt die Kommunikation auf die syntaktischen Ebene und auf die semantischen Ebene kommen Ontologien und Vokabularen zum Einsatz. Das auf einer Nachricht richtig gehandelt/reagiert wird setzt voraus, dass die multiple of Standards in benutzen mit einander "Aligned werden können" <rebstock, S: [5]> dise Tatsache kann für KMUs schwierig sein <rebstock, S: 4>. In sein 2006 erschienen Artikel, hat Tim Berners-Lee das Semantic Web nach <cite> etwas beschiedener formuliert als Linked Data. das Konzept der Linked Data formuliert um ein nach <cite> etwas bescheiden die ursprunglichen Vision des Semantic Webs, etwas bescheidender beschrieben als ein "Web of Data" Nach bestem Wissen wurde es aber bis jetzt noch nicht erforscht ob ein Bottom-Up Ansatz für KMUs die nur gelegentlich mit einander JSON formatierte daten austauschen.  der Einsatz...   im Nach \citeauthor[S.~488]{may2006semantic} kann man Ontologie im Kontext der Semantic Webs wie folgt charakterisieren: \hyphenblockquote{german}{Eine Ontologie beschreibt Wissen über Konzepte und ihre Zusammenhänge so, dass z. B. einerseits eine Klassifizierung eines Objektes anhand dessen Eigenschaften möglich ist, und andererseits aus dem Wissen über die Konzeptzugehörigkeit eines Objektes weitere Schlüsse über das Objekt und Beziehungen zu seiner Umwelt möglich sind.} Obwohl ist nach S.17 kein einheitliche Verwendung des Begriffes Ontologie im Kontext des Semantic Webs gibt, wird der gängige Unterschied (laut W3C Vocabularies) in dieser Arbeit (klassifizierung und seine Beziehungen) benutzt. Dieser hsdf Vision hat sich nict durchgesetzt ist der Fokus zuruckgegangen um 2006 hat Tim Berners-Lee das Konzept von Linked Data introduced. Linked Data beschreibt eine nach \citeauthor[S.~61]{dewilde2015information} bescheidene Version des Semantic Webs (auch manchmal \hyphenquote{german}{Web of Data} genannt) indem Web Konzepte und RDF Standards benutzt werden um disperate Datasets im Netzt mit einander dezentralisiert verknupft werden können.  der sagt eine Maschine kann das Semantic Web (auch Web of Data im Beitrag bezeichnet) und vier Regeln (auch als Erwartungen) gesetzt: 

\begin{enumerate}
	\item Use URIs as names for things
	\item Use HTTP URIs so that people can look up those names.
	\item When someone looks up a URI, provide useful information, using the standards (RDF*, SPARQL)
	\item Include links to other URIs. so that they can discover more things.
\end{enumerate}

Um sicher zu stellen das der Begriff einer URI von einem unfassener "Community" von Anwender gleich ist (Linked Data \hyphenquote{german}{Regel} \#1), ist der Ansatz von Vokabulare (oder Ontologien) essenziell. Es gibt eine umfassende Zahl von Vokabulare die im Web benutzt werden können, die am häufigsten benutzten laut LOD <footnote> sind FOAF, SKOS, Dublincore. Da es eine vielfältige Menge an Vokabulare gibt, es ist wichtig das sie miteinander "Aligned" sind (soweit es geht) um die Interopabilität zu gewähren<cmt: Semiotik Discussion> <cite>. Schema.org ist auch ein Vokabulare die von den große Suchmaschineanbieter zusammenentwickelt wurde und stellt eine vereinfachte Vokabulare dar. <cite: nächste Abbildung> Als Beispiel dazu, hat das Globmaplab\footnote{\url{http://kartenlabor.uni-erfurt.de}} seine Datenbestände weitgehend an das GND-Ontologie "aligned" und mithilfe von RDFa und JSON-LD seine HTML und JSON APIs mit Meta-Daten versehen.se Möglichkeit 
se Möglichkeit 




