\documentclass[12pt,
				a4paper,
				twoside=false,
				titlepage=true,
				bibliography=totoc, %bibl zu InhVz
				listof=totoc, %Verzeichnisse zu InhVz
				numbers=noenddot]{scrartcl}
\usepackage{lmodern}
\usepackage{fontspec}
% used to insert formatted code blocks (for many different programming languages)
\usepackage{minted}
% simply table creation
\usepackage{tabularx}

\usepackage[ngerman]{babel}  %Umlaute mit "erzeugen !!!
\usepackage[colorlinks, linkcolor = black, citecolor = black, filecolor = black, urlcolor = blue]{hyperref}
\usepackage[includeheadfoot,paper=a4paper,left=3.5cm,right=2.5cm,top=1.6cm,bottom = 1.3cm]{geometry}
\usepackage[table,xcdraw]{xcolor}

\newcommand{\changefont}[3]{
\fontfamily{#1} \fontseries{#2} \fontshape{#3} \selectfont}

\usepackage{float}
\floatstyle{boxed} 
%\restylefloat{figure}	%Ränder um alle Abbildungen

\usepackage{fancyhdr} %header
\usepackage{booktabs,ctable} %Tabellen
\usepackage[format=hang,indention=1cm,font=md ,labelfont=bf,
			labelsep=space, %kein ':' hinter Abb., Tab.
			nooneline,
			justification=centering
			]{caption}
\captionsetup{font=small}

%Unterschriften aller Art
\addto\captionsngerman{%
\renewcommand{\figurename}{Abb.}} %Abb. statt Abbildungen
\addto\captionsngerman{%
\renewcommand{\tablename}{Tab.}} %Tab. statt Tabelle
\addto\captionsngerman{%
\def\listingname{Listing}} % für Quellcode
\usepackage{eurosym} %Eurosymbol mit \euro{}
\usepackage{url} %URLSs adden mit \url{}
\usepackage{titlesec} % Textüberschriften anpassen


%%%%%%%%%%%%%%%%%%%%%% titlesec %%%%%%%%%%%%%%%%%%%%%%%%%%%%%

\titleformat{\section}[hang]{\large\bfseries}{\thesection. }{0pt}{}%space in der hinteren oder punkte in der vorderen anders entscheinden
\titleformat{\subsection}[hang]{\normalsize\bfseries}{\thesubsection. }{0pt}{}
\titleformat{\subsubsection}[hang]{\normalsize\bfseries}{\thesubsubsection. }{0pt}{}
\titleformat{\paragraph}[hang]{\normalsize\mdseries\itshape}{\theparagraph\quad}{0pt}{} %mdseries: Nicht fett; itshape: kursiv

\titlespacing{\section}{0pt}{0pt}{8pt}
\titlespacing{\subsection}{0pt}{20pt}{8pt}
\titlespacing{\subsubsection}{0pt}{20pt}{8pt}
\titlespacing{\paragraph}{0pt}{30pt}{3pt}


%%%%%%%%%%%%%%%%%%%%%% titlesec %%%%%%%%%%%%%%%%%%%%%%%%%%%%%

\usepackage[autostyle,         						
	 		%german=guillemets,
	  		]{csquotes}
	  		
\usepackage[backend=biber,
       style=authoryear-icomp-wi,
       citestyle=authoryear-icomp-wi,
       bibstyle=authoryear-icomp-wi,
       firstinits=true,
       hyperref=true,
       ibidtracker=false,
       terseinits=true,
       uniquename=false,
       uniquelist=false,
       sortlocale=de_DE,
       pagetracker=false,
       natbib=true,
       url=false, 
       doi=false,
       isbn=false,
       eprint=false,
       maxcitenames=2,
       maxnames=2,
       minnames=1,
       maxbibnames=99,
       sorting=nyt
       ]{biblatex}
       
\setcounter{biburllcpenalty}{7000}		%Kürzen von Überlangen URLs in dem Literaturverzeichnis
	
\usepackage{tocloft}
\renewcommand\cftfigpresnum{\bfseries Abb. }
\renewcommand\cftfigaftersnum{\normalfont}\settowidth{\cftfignumwidth}{\cftfigpresnum}
\renewcommand{\cftfigaftersnumb}{\quad}		%Abb. vor jeder Abbildung im Abbildungsverzeichnis + Nummerierung

\renewcommand\cfttabpresnum{\bfseries Tab. }
\renewcommand\cfttabaftersnum{\normalfont}\settowidth{\cfttabnumwidth}{\cfttabpresnum}
\renewcommand{\cfttabaftersnumb}{\quad}		%Tab. vor jeder Tabelle im Tabellenverzeichnis + Nummerierung
	
	
	    		
\usepackage{graphicx} %Bilder Einfügen
\usepackage[onehalfspacing]{setspace} %1,5 Zeilenabstand
\usepackage[printonlyused]{acronym} %Abkürzungsverzeichnis

\setlength{\bibitemsep}{1em}     % Abstand zwischen den Literaturangaben
\setlength{\bibhang}{2em}        % Einzug nach jeweils erster Zeile
\addbibresource{Literatur.bib}
\DeclareLanguageMapping{ngerman}{ngerman-apa} 						
\defbibheading{head}{\section*{Literaturverzeichnis}}
\setkomafont{sectioning}{\normalcolor\bfseries}
\urlstyle{same}

%%%%%%%%%%%%%%%%%%%%%%%%%% Formelverzeichnis %%%%%%%%%%%%%%%%
\DeclareNewTOC[%
 type=formel,
 name={Formel},%
 hang=1.9em,%
 listname={Formelverzeichnis}
]{for}

\newcommand*{\formelentry}[1]{%
 \addcontentsline{for}{formel}{\textbf{Form.} \protect\numberline{\textbf{\theequation}} #1}%
} %Abkürzung Form. vor jeder Formel im Formelverzeichnis


%%%%%%%%%%%%%%%%%%%%%%%%%% Formelverzeichnis %%%%%%%%%%%%%%%%

%%%%%%%%%%%%%%%%%%%%%%%%%% Header %%%%%%%%%%%%%%%%%%%%%%%%%%%

\pagestyle{fancy}
\fancyhf{}
\fancyfoot[R]{\thepage}
\fancyhead[R]{}
\fancyhead[L]{RDF basierte semantische Annotationsansätze}
\usepackage{blindtext}


%%%%%%%%%%%%%%%%%%%%%%%%%%%%%%%%%%%%%%%%%%%%%%%%%%%%%%%%%%%%%%%%%%%%%
%%%%%%%%%%%%%%%%%%  Beginn des Dokuments %%%%%%%%%%%%%%%%%%%%%%%%%%%
%%%%%%%%%%%%%%%%%%%%%%%%%%%%%%%%%%%%%%%%%%%%%%%%%%%%%%%%%%%%%%%%%%%%%
\usepackage[nosumlimits]{amsmath}
\usepackage{amssymb}
\usepackage{multirow}
\usepackage{here}
\usepackage{makeidx} %Indexerstellung
\makeindex %Ausführung der Indexerstellung
\usepackage[german, intoc]{nomencl} %Für Symbolvverzeichnis (german: Deutsche Benamung; refeq: siehe Gleichung <Nr.>; refpage:Seite <Nr.>; intoc: Aufnahme in Inhaltsverzeichnis)
\makenomenclature %Ausführung der Erstellung des Symbolverzeichnisses
\parindent 0pt %Setzt Einrückung beim Absatz auf 0pt
\parskip .75em %Setzt Abstand zwischen Absätzen auf .75em
\begin{document}


%%%%%%%%%%%%%%%%%%%%%%% Titelseite %%%%%%%%%%%%%%%%%%%%%%%%%%%

\begin{titlepage}
\begin{minipage}{1\textwidth}
\flushright 
   \includegraphics[height=1.7cm]{Logo_schwarzgruen_04}
\end{minipage}

\vspace{3cm}
\begin{center}
\textbf{\LARGE{RDF basierte semantische Annotationsansätze}}
\\
%\textbf{Untertitel}
\end{center}
\vspace{1.5cm}

\begin{center}
\textbf{Art der Arbeit}
\end{center}

\vspace{3.5cm}

\begin{flushleft}
\begin{tabular}{lll}
\textbf{Betreuer:} & &  ...\\
& & \\
\textbf{vorgelegt von:}& & ...\\
& & Straße\\
& & PLZ Ort\\
& & Telefonnummer\\
& & E-Mail\\
& & \\
\textbf{Matrikelnummer:} & & ...\\
& & \\

\textbf{Bearbeitungszeitraum:} & & ... - ....\\
& & \\

\textbf{Abgabetermin:} & & ...
\end{tabular}
\end{flushleft}
\end{titlepage}


%%%%%%%%%%%%%%%%%%%%%% Verzeichnisse %%%%%%%%%%%%%%%%%%%%%%%%%
\pagenumbering{Roman}\setcounter{page}{2}
\newpage
\tableofcontents % Inhaltsverzeichnis
\newpage
\listoffigures % Abbildungsverzeichnis
\newpage 
\listoftables % Tabellenverzeichnis 
\newpage
\listofformels %Formelverzeichnis einfügen
\newpage
\section*{Abkürzungsverzeichnis} 
\addcontentsline{toc}{section}{Abkürzungsverzeichnis}

\begin{acronym}[TDMA]
	\acro{Abk.}{Abkürzung}
\end{acronym}

\newpage %Abkürzungsverzeichnis
\newpage

%%%%%%%%%%%%%%%%%%%  Beginn des Textes  %%%%%%%%%%%%%%%%%%%%%

\pagenumbering{arabic}

%% INCLUDES:

\section{Einleitung}
\subsection{Motivation}
Im Zuge der Globalisierung sind viele Unternehmen vom hohen Wettbewerbsdruck betroffen <todo: "Handbuch Interorganisationssysteme noch zitieren">. Um ihren Ertrag langfristig zu sichern und innovativ gegenüber der Konkurrenz zu bleiben, bieten viele Industrieunternehmen ,,Value Added Services``, um die steigende Dienstleistungsnachfrage ihrer Kunden zu befriedigen \citep[vgl.][S. 4]{Meffert2015}. Das Erbringen dieser Dienstleistungen erfordert Geschäftsprozesse die zwischenbetrieblich ausgerichtet sind \parencite[vgl.][S.19]{fleisch2001netzwerkunternehmen}. Im Kontext einer vernetzten Welt \parencite[vgl.]{bmwi2013} ist es von besonderer Bedeutung, dass zwischenbetriebliche Daten- und Funktionsintegration auch erfolgen kann, wenn Daten über das World Wide Web ausgetauscht werden. 

\subsection{Methodik}
Das Ziel der vorliegenden Arbeit ist ein Literatur-Review durchzuführen, um die relevante Literatur über das Resource Description Framework auszuwählen, zu analysieren und zusammenzufassen im Hinblick auf den möglichen Einsatz von RDF für semantische Annotation von Daten innerhalb kleiner und mittelständischer Unternehmen. Das Review wird anhand der Methodik von \cite{fettke2006state} durchgeführt. Eine strukturierte Literatursuche wurde mithilfe der Richtlinien von \cite[S. XVI]{webster2002analyzing} umgesetzt, um eine Liste von relevanten Literatur zu erstellen. 
\begin{enumerate}
	\item Im ersten Schritt wurden \emph{EBSCOHost}, \emph{Google Scholar} und \emph{W3C Recommendations} verwendet, um eine Literaturliste zu erstellen.
	\item Danach wurden alle Quellen der Literaturliste untersucht, um weitere relevante Literatur zu finden (sogenannte \hyphenquote{german}{Go Backward} Ansatz).
	\item Schließlich wurden Google-Scholar und Semantic-Scholar benutzt, um weitere Publikationen zu identifizieren, die auf der Literaturliste von Schritt eins und zwei verweisen (sogenannte \hyphenquote{german}{Go Forward} Ansatz).
\end{enumerate}
Die folgenden Suchschlüsseln wurden während der Literatursuche benutzt:
\begin{itemize}
	\item Primärschlüssel: \texttt{Resource Description Framework, RDF, JSON-LD, RDFa}
	\item Sekundärschlüssel: \texttt{semantic annotation, semantic web, linked data}
\end{itemize}
Zum Schluss wurde die Liste der relevanten Literatur mit Blick auf der Zielstellung verarbeitet und nach wichtige Konzepte systematisiert \parencite[vgl.][S. XVI]{webster2002analyzing}.

\subsection{Aufbau}
Nach der Einleitung wird die Entwicklung von RDF im zweiten Abschnitt kurz beschrieben, und die benötigte RDF-Grundlagen und zusammenhängende Begrifflichkeiten werden herausgearbeitet. Die Syntax von RDF wird auch anhand einiger Beispiele demonstriert. Das Konzept des Semantic Webs und der Linked Data wird im dritten Abschnitt vorgestellt. Das Resource Description Framework in Attributes (RDFa) wird als Serialisierungssyntax ausgewählt um die Konzepte der semantischen Annotation im Web of Data exemplarisch zu zeigen. Im letzten Abschnitt werden die Kernkonzepte der Arbeit zusammengefasst und es wird beschrieben, welche Weiterentwicklungen künftig von Interesse sein könnten.
\section{RDF Grundlagen}
\subsection{Entwicklung}
\label{sec:entwicklung}

Das Resource Description Framework (RDF) wurde ursprünglich 1999 vom World Wide Web Consortium (W3C) als Empfehlung verabschiedet \parencite{Lassila:99:RMS}. 2004 wurde diese Version aktualisiert und in mehrere Dokumente aufgeteilt \parencite{Beckett:04:RSS}. Die aktuelle, erweiterte Version (RDF 1.1) wurde 2014 veröffentlicht \parencite{Schreiber:14:RP}. RDF 1.1, im Weiteren nur als RDF bezeichnet sofern nicht anders festgelegt, hat das Ziel die Unterstützung neuer Anwendungsfelder für RDF zu stärken. \autoref{tab:rdf} zeigt die Erweiterung von RDF 1.0 auf RDF 1.1 (vgl. \cite[Abs.~2]{Klyne:04:RDF}; \cite[Abs.~2]{Schreiber:14:RP}; \cite{Wood:14:WNR}). Davon kann man ableiten, dass RDF sich in die Richtung entwickelt, immer mehr domänenspezifische Anwendung zu unterstützen. Das bedeutet, dass kleinere Unternehmen mit vereinfachte JSON basierte REST-APIs Datenintegration\footnotemark{} auf die syntaktischen und semantischen Ebene anhand RDF wirksam einsetzen können.
\footnotetext{vgl. die semiotische Ebenen der Integration in \citeauthor{Schissler2004}}

\begin{table}[h]
	\centering
	\begin{tabular}{|p{9em}|c|c|}
		\hline \rule[-2ex]{0pt}{5.5ex} Anwendungsfall & RDF 1.0 & RDF 1.1 \\ 
		\hline \rule[-2ex]{0pt}{5.5ex} Web-Metadaten & RDF/XML & HTML5+RDFa 1.1, JSON-LD\\ 
		\hline \rule[-2ex]{0pt}{5.5ex} Datenaustausch zw. Datenbanken\footnotemark{}& RDF/XML & JSON-LD, TriG, N-Quads  \\
		\hline \rule[-2ex]{0pt}{5.5ex} API-Feeds Verbinden \footnotemark[\value{footnote}] & RDF/XML& JSON-LD, RDF/XML\\ 
		\hline
	\end{tabular}
	\caption{Die Entwicklung von RDF}
	\label{tab:rdf}
\end{table}
\footnotetext{vgl. \cite[Abs.~2]{Klyne:04:RDF}; \cite[Abs.~2]{Schreiber:14:RP}; \cite{Wood:14:WNR}}

\subsection{Beschreibung} 

Laut \citeauthor{Schreiber:14:RP} in \citetitle{Schreiber:14:RP}: 

\hyphenblockquote{german}{RDF is intended for situations in which information on the Web needs to be processed by applications, rather than being only displayed to people. RDF provides a common framework for expressing this information so it can be exchanged between applications without loss of meaning.} 

Wie der Name vermuten lässt, bietet das Resource Description Framework ein Gerüst (Modell, Sprachen und Syntaxen) für die Beschreibung von Attributen, Funktionen und Beziehungen der Ressourcen. Ressourcen können alles sein, was einen einzigartigen Identifier (URI oder IRI) hat \parencite[vgl.][Folie~6]{Dekeyzer2013}. RDF legt eine abstrakte Syntax fest um Zusammenhänge zwischen Ressourcen als eine Menge gerichteter Graphen darzustellen. Ein gerichteter Graph (genannt \hyphenquote{german}{Triple} in RDF) hat eine Knote (Subjekt), die über eine gerichtete Kante (Prädikat) mit einer anderen Knote (Objekt) in Verbindung steht. \autoref{fig:rdf-intro} veranschaulicht diesen Aspekt. Das Beispiel\footnotemark{} drückt die Beziehung zwischen einem Kartograph und seinen produzierten Werken aus. 

\footnotetext{Das Datenmodell basiert auf einer Web-Applikation (\href{http://kartenlabor.uni-erfurt.de}{Globmaplab}), die für die Arbeit mit den historischen Beständen der Sammlung Perthes konzipiert wurde. Offenlegung: der Autor dieser Arbeit war Lead-Entwickler dieses Projekts.}

\begin{figure}[h]
	\centering
	\includegraphics[width=1\linewidth]{images/rdf-intro}
	\caption[Kernkonzepte des RDFs]{Kernkonzepte des RDFs anhand eines Beispielgraphs}
	\label{fig:rdf-intro}
u\end{figure}

Die konkrete RDF-Serialisierungsyntaxen werden im \autoref{sec:serialisierung} erörtert. Jeder Komponent eines RDF-Tripels kann ein von drei Ausprägungen haben \parencite[vgl.][Abs.~3.1]{Wood:14:RCA}.

\begin{itemize}
	\item Ein Subjekt ist ein \textit{IRI} oder ein \textit{Blank Node}.
	\item Ein Prädikat ist ein \textit{IRI}.
	\item Ein Objekt ist entweder ein \textit{IRI}, ein \textit{Literale} oder ein \textit{Blank Node}.
\end{itemize}

Ein \textit{IRI} (International Resource Identifier, festgelegt in RFC 3987) ist eine Generalisierung eines URIs (Uniform Resource Indicator, RFC 3986), die Zeichen aus der nicht-ASCII Bereich der Universal Character Set\footnotemark{} erlauben (vgl. \cite[][Abs~3.2]{Schreiber:14:RP}; \cite[]{rfc3987}). IRIs können die Eigenschaften haben, durch die Internet und World Wide Web (TCP/IP + DNS + HTTP) global eindeutig und \hyphenquote{german}{dereferenzierbar} zu sein \parencite[vgl.][Abs.~2]{Jacobs:04:AWW}. 

Ein \textit{Literal} ist eine konkrete Ausprägung eines Datentyp (wie z. B. eine Zeichenkette, Zahl oder Datum) und kein IRI. Ein String-Literal kann wahlweise mit einem \hyphenquote{german}{language tag} assoziiert sein um die beinhaltete Sprache zu kennzeichnen. RDF Literals können alle Datentypen, die in der XML Schema Definition Language\footnote{\url{http://www.w3.org/TR/xmlschema11-2/}} definiert sind, verwenden \parencite[vgl.][Abs. 5]{Wood:14:RCA}. 

\textit{Blank Nodes} sind von IRIs und Literals disjunkt. Diese Abgrenzung macht es möglich, Ressourcen die keine URI haben in RDF abzubilden. 

\footnotetext{wie in Norm ISO/IEC 10646 und in \hyphenquote{german}{\href{http://www.unicode.org/versions/latest}{The Unicode Standard}} vorgegeben ist}

\subsubsection{Serialisierung}
\label{sec:serialisierung}
Mit der Freigabe von RDF 1.1 wurden vier nicht XML-basierten Serialisierungssyntaxen eingearbeitet \parencite[vgl.][Abs.~3]{Wood:14:WNR}. In dieser Arbeit werden Resource Description Framework in Attributes (RDFa) Syntax und JSON Linked Data (JSON-LD) behandelt. Die beiden Serialisierungssyntaxen ermöglichen semantische Annotation (vgl. \autoref{sec:linked-data} für Definition) in Anwendungsfelder wo es vorher mit klassichen RDF/XML nicht ideal war. Semantische Annotation ist nach \citeauthor[S.~405f]{reif2006semantic}, \hyphenquote{german}{den Prozess des Hinzufügens von semantischen Meta-Daten zu Dokumenten, die den Inhalt eines Dokuments in maschinen-verarbeitbarer Form beschreiben}.

RDFa macht es möglich maschinenlesbare Metadaten in Web-Seiten einzubinden indem es neue HTML-Attributen für diesen Zweck festlegt. Diese \hyphenquote{german}{Anreicherung} der Metadaten einer Website führt dazu, dass Fremdsoftware in der Lage sind die Metadaten der Webseite automatisch verarbeiten zu können, und dass Suchmaschinen eine gezielter Darstellung des Websiteinhalts für Suchergebnisse anbieten können \parencite[vgl.][Abs.~2]{Schreiber:14:RP}, wobei der letztere Punkt insbesondere dann der Fall ist, wenn die Metadaten an weit verbreitete Ontologien und Vokabulare angeglichen werden (vgl. \autoref{sec:linked-data}). Eine RDFa Serialiserungsmöglichkeit für das Datenmodell in \autoref{fig:rdf-intro} wurde in \autoref{lst:rdfa} veranschaulicht.

\begin{listing}[H]
\begin{minted}[linenos,
numbersep=5pt,
frame=single,
framesep=2mm]{html}
<html>
<body>
<p>test</p>
</body>
</html>
\end{minted}
\caption{Datenmodell in RDFa}
\label{lst:rdfa}
\end{listing}

JSON-LD wurde ursprünglich entwickelt mit der Absicht die Interoperabilität von Webservices für Linked Data (vgl. Definition in \autoref{sec:linked-data}) auszubauen und Linked Data in JSON-basierte Datenbanksysteme abzuspeichern \parencite[vgl.][Abs.~1]{Lanthaler:14:J}.\footnote{\cite{Vincent2015} zeigt auch wie JSON-LD benutzt werden kann um Webseiteninhalt semantisch zu annotieren.} Die Möglichkeit Linked Data (genauer gesagt RDF-Graphen) in der Javascript Object Notation\footnote{\cite[vgl.]{ecma2013}} (JSON) zu serialisieren lässt eine reibungslose Upgrade-Pfad für eingesetzte Systeme zu. Beispiel X veranschaulicht wie die existierende JSON-basierte REST-APIs des Globmaplabs mit JSON-LD als RDF-Graphen ausgedruckt werden können.
\section{Linked Data und Serialisierung}
\label{sec:linked-data-serialisierung}

\subsection{Semantic Web}
\label{sec:semantic-web}

Einer der Vorreiter des World Wide Webs Tim Berners-Lee hat zusammen mit anderen Autoren \citeyear{berners2001semantic} ein Artikel im Scientific American Journal mit dem Titel, \citetitle{berners2001semantic} publiziert. In diesem Artikel haben die Autoren der Konzept des Semantic Webs eingeführt und eine Vision der Zukunft formuliert: \hyphenquote{german}{The Semantic Web will bring structure to the meaningful content of Web pages, creating an environment where software agents roaming from page to page can readily carry out sophisticated tasks for users.}

Um diese Vision gerecht zu werden ist nach \citeauthor{blumauer2006semantic} der Begriff \hyphenquote{german}{Semantic Web} genauer als \hyphenquote{german}{Semiotic Web} zu verstehen. Aus Sicht der Semiotik setzt dieser Art von Interoperabilität zwischen Akteure im Semantic Web voraus, dass sie sich auf die syntaktischen, semantischen und pragmatischen Ebene verständigen können. Das heißt, wenn ein Sender eine Nachricht zum Empfänger Schickt, ist der Empfänger in der Lage die Nachricht richtig zu lesen (Syntax), zu interpretieren (Semantik), und schließlich richtig darauf zu reagieren (Pragmatik)\parencite[vgl.]{voigtmann2002enterprise}. 

RDF unterstützt die Kommunikation auf die syntaktischen Ebene und auf die semantischen Ebene kommen Ontologien zum Einsatz. Laut \citeauthor[S.~488]{may2006semantic} kann man Ontologie im Kontext des Semantic Webs wie folgt charakterisieren: \hyphenblockquote{german}{Eine Ontologie beschreibt Wissen über Konzepte und ihre Zusammenhänge so, dass z. B. einerseits eine Klassifizierung eines Objektes anhand dessen Eigenschaften möglich ist, und andererseits aus dem Wissen über die Konzeptzugehörigkeit eines Objektes weitere Schlüsse über das Objekt und Beziehungen zu seiner Umwelt möglich sind.} Erst wenn geeignete Ontologien existieren und die beteiligte Akteure sich an festgelegte Standards halten ist es möglich auf einer Nachricht richtig zu reagieren (Pragmatik). Dennoch ist Integration auf diesen pragmatischen Ebene aufgrund der stetigen, wachsenden Anzahl an Standards (z. B. ebXML, RosettaNet, Biztalk, etc.) und häufig ändernde Geschäftsprozesse in der globalisierten Wirtschaft für viele KMUs mit zu hohen Kosten verbunden \parencite[vgl.][S.~4ff]{rebstock2008ontologies}.

\subsection{Linked Data}
\label{sec:linked-data}

In sein 2006 erschienen Artikel hat Tim Berners-Lee das Semantic Web, nach \citeauthor{dewilde2015information}, etwas beschiedener formuliert als Linked Data. Linked Data zielt darauf hin, ein \hyphenquote{german}{Web Of Data} zu schaffen indem es vier Prinzipien festlegt um Inhaltsinhaber zu ermuntern, ihre Datensätze untereinander im World Wide Web zu verlinken \parencite{berners2006linked}: 

\begin{enumerate}
	\item Use URIs as names for things
	\item Use HTTP URIs so that people can look up those names.
	\item When someone looks up a URI, provide useful information, using the standards (RDF*, SPARQL)
	\item Include links to other URIs. so that they can discover more things.
\end{enumerate}

\autoref{fig:rdf-intro} zeigt eine erweiterte Darstellung des Datenmodells der ersten Abbildung. Der lila RDF Graph links, der Justus Perthes abbildet, und der grün RDF Graph rechts, der die gezeichnete Karte abbildet, nutzen URIs von bekannten Ontologien, die die vier Prinzipien der Linked Data implementieren.

\begin{figure}[h]
	\centering
	\includegraphics[width=1\linewidth]{images/rdf-intro}
	\caption[Linked Data mit RDF]{Linked Data mit RDF}
	\label{fig:rdf-intro}
\end{figure}

\subsubsection{Serialisierung}
\label{sec:serialisierung}

Mit der Freigabe von RDF 1.1 wurden vier nicht XML-basierten Serialisierungssyntaxen eingearbeitet \parencite[vgl.][Abs.~3]{Wood:14:WNR}. In dieser Arbeit werden Resource Description Framework in Attributes (RDFa) Syntax und JSON Linked Data (JSON-LD) behandelt. Die beiden Serialisierungssyntaxen ermöglichen semantische Annotation (vgl. \autoref{sec:linked-data} für Definition) in Anwendungsfelder wo es vorher mit klassichen RDF/XML nicht ideal war. Semantische Annotation ist nach \citeauthor[S.~405f]{reif2006semantic}, \hyphenquote{german}{den Prozess des Hinzufügens von semantischen Meta-Daten zu Dokumenten, die den Inhalt eines Dokuments in maschinen-verarbeitbarer Form beschreiben}.

RDFa macht es möglich maschinenlesbare Metadaten in Webseiten einzubinden indem es neue HTML-Attributen für diesen Zweck festlegt. Diese \hyphenquote{german}{Anreicherung} der Metadaten einer Website führt dazu, dass Fremdsoftware in der Lage sind die Metadaten der Webseite automatisch verarbeiten zu können, und dass Suchmaschinen eine gezielter Darstellung des Websiteinhalts für Suchergebnisse anbieten können \parencite[vgl.][Abs.~2]{Schreiber:14:RP}, wobei der letztere Punkt insbesondere dann der Fall ist, wenn die Metadaten an weit verbreitete Ontologien und Vokabulare angeglichen werden. Eine RDFa Serialisierungsmöglichkeit für das Datenmodell in \autoref{fig:html-map} wurde in \autoref{lst:rdfa} veranschaulicht.

\begin{figure}[h]
	\centering
	\includegraphics[width=.7\linewidth]{images/htmlmap_marked}
	\caption[Karte und Zugehörige Autoren]{Karte und Zugehörige Autoren (Screenshot vom Globmaplab Webseite)}
	\label{fig:html-map}
\end{figure}

\begin{listing}[H]
\begin{minted}[linenos,
numbersep=5pt,
frame=single,
framesep=1mm]{html}
<body prefix="gndo: http://d-nb.info/standards/elementset/gnd#
      rdau: http://rdaregistry.info/Elements/u/"
>
...
<div resource="http://kartenlabor.uni-erfurt.de/maps/5"
     typeof="http://schema.org/Map">
  <h4 property="gndo:preferredNameForTheWork">Süd-America</h4>	
  <dt>Bezeichner:</dt>
  <dd>5</dd>
  ...
</div>
...
<h3>Dazugehörige Personen</h3>
  <table>
    <tbody>
      ...
      <div resource="http://d-nb.info/gnd/116091991"
           typeof="gndo:cartographer">
        <tr property="rdau:cartographerOf" 
            resource="http://kartenlabor.uni-erfurt.de/maps/5">
          <td>
            Perthes, Justus ( * 1749-9-11 † 1816-5-1 ) 
            as Cartographer
          </td>
        </tr>
      </div>
      ...
\end{minted}
\caption{Datenmodell in RDFa}
\label{lst:rdfa}
\end{listing}

Auf die erste Zeile im \autoref{lst:rdfa} sieht man das \texttt{prefix} Attribut, das zwei Abkürzungen zu bekannten Ontologien festlegt. Das Attribut \texttt{gndo} verweist auf die Gemeinsame Normdatei (GND) Ontologie der Deutsche National Bibliothek und \texttt{rdau} auf das Resource Description \& Access (RDA) Vokabular.  Zeile 5 bis 10 bildet das HTML Definition List in der rechten Hälfte der \autoref{fig:html-map} ab. Mittels RDFa kann man zum Ausdruck bringen, dass die Definitionsliste eine Karte (\texttt{typeof="http://schema.org/Map"}) mit einem IRI Bezeichner innerhalb die Uni-Erfurt Domain und dem Namen \hyphenquote{german}{Süd-America} beschreibt. Zeile 13 bis 29 stellt die Tabelle unten in \autoref{fig:html-map} dar. Hier wird die Ressource Justus Perthes (\texttt{http://d-nb.info/gnd/116091991}) mit der Süd-America Karte verlinkt, indem er die Rolle des Kartographs übernimmt. 
\section{Schlussbemerkungen}
\subsection{Zusammenfassung}

Die Weiterentwicklung von RDF in den letzten zehn Jahren hat den Ansatz von RDF in einer Vielfalt von Anwendungsszenarien ermöglicht. Ein mögliches Anwendungsszenario, nämlich das Hinzufügen von Metadaten im Webseiten mittels RDFa, wurde durch faktische Datensätze von das Globmaplab Web-Applikation erörtert. Dabei wurden die vier Prinzipien der Linked Data demonstriert. Abschließend sollte diese Arbeit den Leser die grundlegendene Konzepte der RDF basierten semantischen Annotationssprachen vermittelt haben.

\subsection{Kritische Würdigung}
Aufgrund des Zeitmangels war es nicht möglich Vergleiche zwischen den neuen Serialisierungssyntaxen von RDF 1.1, wie zum Beispiel zwischen JSON-LD und RDFa, herzustellen oder Anwendungsansätze mit JSON-LD zu zeigen. Darüber hinaus wurde das dritte Prinzip der Linked Data \hyphenquote{german}{When someone looks up a IRI, provide useful information, using the standards (RDF*, SPARQL)} strenggenommen im \autoref{fig:rdf-intro} verletzt. Die Prädikat zwischen Justus Perthes und der Karte von Sud-Amerika drückt eine \hyphenquote{german}{ist Kartograph von} Beziehung aus, aber der IRI \texttt{http://rdaregistry.info/Elements/u/cartographerOf} ist tatsächlich nicht dereferenzierbar da es nur einen lexikalischen Alias des kanonischen IRIs\footnote{\url{http://rdaregistry.info/Elements/u/P60654}} ist. Aus illustrative Gründen wurde aber der lexikalische IRI benutzt. 

\subsection{Ausblick}
Wie im Abschnitt \hyphenquote{german}{Kritische Würdigung} erwähnt wurde, war es nicht möglich JSON-LD basierte semantische Annotationsansätze zu zeigen. Nach bestem Wissen wurde es bisher noch nicht erforscht ob der Ansatz von JSON-LD die überbetriebliche Datenintegration zwischen KMUs, die JSON basierte REST-APIs über XML-basierte Webservices bevorzugen, unterstützen könnten\footnotemark{}. Ein solcher Ansatz von JSON-LD könnte in der Zukunft eine vielversprechende Entwicklung sein.
\footnotetext{Vgl. \cite{Benslimane2008} weshalb solcher Ansatz für KMUs von Interesse sein könnten. }
%%

\pagenumbering{Roman}\setcounter{page}{9}

\addcontentsline{toc}{section}{Literaturverzeichnis}
\printbibliography[heading=head]

%%%%%%%%%%%%%%% Eidesstattliche Erklärung %%%%%%%%%%%%%
\addcontentsline{toc}{section}{Eidesstattliche Erklärung} %Abkvz in TOC
\renewcommand\refname{Eidesstattliche Erklärung} \section*{Eidesstattliche Erklärung}


\vspace*{5em}
\normalsize
Ich versichere an Eides statt durch meine Unterschrift, dass ich die vorstehende Arbeit selbstständig und ohne fremde Hilfe angefertigt und alle Stellen, die ich wörtlich oder annähernd wörtlich aus Veröffentlichungen entnommen habe, als solche kenntlich gemacht habe, mich auch keiner anderen als der angegebenen Literatur oder sonstiger Hilfsmittel bedient habe. Die Arbeit hat in dieser oder ähnlicher Form noch keiner anderen Prüfungsbehörde vorgelegen.
  \vspace*{5em}
\\
Ilmenau, \ Tag. Monat Jahr \ \ \ \underline{\ \ \ \ \ \ \ \ \ \ \ \ \ \ \ \ \ \
     \ \ \ \ \ \ \ \ \ \ }\\
\hspace*{12.6em}\small{\ Unterschrift}

\end{document}

%folgenden Befehl per cmd ausführen um ein Formelverzeichnis zu erstellen
%makeindex Arbeit.nlo -s nomencl.ist -o Arbeit.nls