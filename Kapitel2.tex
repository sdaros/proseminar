\section{RDF Grundlagen}
\subsection{Entwicklung}
Das Resource Description Framework (RDF) wurde ursprünglich 1999 vom World Wide Web Consortium (W3C) als Empfehlung verabschiedet \parencite{Lassila:99:RMS}. 2004 wurde diese Version aktualisiert und in mehrere Dokumente geteilt \parencite{Beckett:04:RSS}. Die aktuelle, erweiterte Version (RDF 1.1) wurde 2014 veröffentlicht \parencite{Schreiber:14:RP}. RDF 1.1, im weiteren nur als RDF bezeichnet sofern nicht anders festgelegt ist, hat das Ziel die Unterstützung neuer Anwendungsfelder für RDF zu stärken. Tabelle~\ref{tab:rdf} zeigt die Erweiterung von RDF 1.0 auf RDF 1.1. Davon kann man ableiten, dass RDF sich in der Richtung orientiert, immer mehr domänenspezifische Anwendung zu unterstützen.
\begin{table}[h]
	\centering
	\begin{tabular}{|p{9em}|c|c|}
		\hline \rule[-2ex]{0pt}{5.5ex} Unterstützt & RDF 1.0 & RDF 1.1 \\ 
		\hline \rule[-2ex]{0pt}{5.5ex} Web-Metadaten & RDF/XML & HTML5+RDFa 1.1, JSON-LD\\ 
		\hline \rule[-2ex]{0pt}{5.5ex} Datenaustausch zw. Datenbanken & RDF/XML\footnotemark & JSON-LD, TriG, N-Quads  \\
		\hline \rule[-2ex]{0pt}{5.5ex} API-Feeds Verbinden & RDF/XML\footnotemark[\value{footnote}] & JSON-LD, RDF/XML\\ 
		\hline
	\end{tabular}
	\caption{Die Entwicklung von RDF\\ (vgl. \cite[Abs.~2]{Klyne:04:RDF}, \cite[Abs.~2]{Schreiber:14:RP}, \cite{Wood:14:WNR})}
	\label{tab:rdf}
\end{table}
\footnotetext{Diese Anwendungsfelder wurden in \cite[Abs.~2]{Klyne:04:RDF} nicht expliziert gegeben, aber RDF/XML in diesem Kontext zu nutzen ist auch möglich.}

\subsection{Beschreibung} 
Laut \citeauthor{Schreiber:14:RP} in \citetitle{Schreiber:14:RP}: \hyphenblockquote{german}{RDF is intended for situations in which information on the Web needs to be processed by applications, rather than being only displayed to people. RDF provides a common framework for expressing this information so it can be exchanged between applications without loss of meaning.} Wie die Name schon vermuten lässt, bietet das Resource Description Framework ein Gerüst (Modell, Sprachen und Syntaxen) für die Beschreibung von Attributen, Funktionen und Beziehungen der Ressourcen, wobei Ressourcen können alles was einen einzigartigen Identifier (URI oder IRI) haben kann sein \parencite[vgl.][Folie~6]{Dekeyzer2013}. Im Hinblick auf die Semiotische Ebenen der Integration \parencite[vgl.]{Schissler2004}, können ein Sender und Empfänger mittels RDF eine gemeinsame Sprache haben, um sich auf der syntaktische, und mithilfe von Ontologien und Vokabularen (vgl. Abschnitt~\ref{sec:semantik}), semantische Ebene verständigen zu können. 

RDF stellt Zusammenhänge zwischen Ressourcen als eine Menge gerichteter Graphen dar. Ein gerichteter Graph (genannt \hyphenquote{german}{Tripel} in RDF) hat eine Knote (Subjekt), die über eine gerichtete Kante (Prädikat) mit einer anderen Knote (Objekt) in Verbindung steht. Abbildung \ref{fig:rdf-intro} veranschaulicht diesen Aspekt. Das Beispiel\footnotemark~drückt die Beziehung zwischen einem Kartograph und seinen produzierten Werken in der RDF-abstrakt Syntax aus.

\begin{figure}[h]
	\centering
	\includegraphics[width=0.7\linewidth]{images/rdf-intro}
	\caption[Kernkonzepte des RDFs]{Kernkonzepte des RDFs anhand eines Beispielgraphs}
	\label{fig:rdf-intro}
\end{figure}

Abbildung \ref{fig:rdf-intro} schildert auch, als Orientierung, die Ähnlichkeiten zwischen den Subjekt-Prädikat-Objekt Denkweise von RDF und die typische Objekt-Orientierte Denkweise. Die konkrete RDF-Serialisierungsyntaxen werden im Abschnitt~\ref{sec:syntax} erörtert. Jeder Komponent eines RDF-Tripels kann ein von drei Ausprägungen haben \parencite[vgl.][Abs.~3,1]{Wood:14:RCA}.

\footnotetext{Das Datenmodell basiert auf einer Web-Applikation (\href{http://kartenlabor.uni-erfurt.de}{Globmaplab}), die für die Arbeit mit den historischen Beständen der Sammlung Perthes konzipiert wurde.}

\begin{itemize}
	\item Ein Subjekt ist ein \textit{IRI} oder ein \textit{Blank Node}.
	\item Ein Prädikat ist ein \textit{IRI}.
	\item Ein Objekt ist entweder ein \textit{IRI}, ein \textit{Literale} oder ein \textit{Blank Node}.
\end{itemize}
ein International Resource Identifier (IRI) ist eine generalisierung von URI (als definiert in)...
ein Literal ist
Blank node leiht die Fähigkeit Subjekt nicht definieren zu müssen was anwendung in RDFa Ansätze findet (siehe \ref{empty})

\subsubsection{Syntax}
\label{sec:syntax}
Mit der Freigabe von RDF 1.1 wurden, unter anderem, zwei nicht XML-basierten Serialisierungssyntaxen eingearbeitet \parencite[vgl.][Abs. 3]{Wood:14:WNR}. Die Resource-Description-Framework-in-Attributes (RDFa) Syntax und JSON-Linked-Data (JSON-LD) Syntaxen ermöglichen semantische Annotation (vgl. Abschnitt \ref{sec:linked-data} für Definition) in Anwendungsfelder wo es vorher mit klassichen RDF/XML nicht geeignet war. 

RDFa macht es möglich Maschine-lesbare Metadaten in Web-Seiten einzubinden indem es neue HTML-Attributen für diesen Zweck festlegt. Diese \hyphenquote{german}{Anreicherung} der Metadaten einer Website führt dazu, dass Fremdsoftware (english: Third Party Applications) in der Lage sind die Metadaten der Webseite automatisch verarbeiten zu können, und dass Suchmaschinen eine gezielter Darstellung des Websiteinhalts für Suchergebnisse anbieten können \parencite[vgl.][Abs. 2]{Schreiber:14:RP}, wobei der letztere Punkt insbesondere dann der Fall ist, wenn die Metadaten an weit verbreitete Ontologien und Vokabulare angeglichen werden (vgl. Abschnitt~\ref{sec:semantik}). Eine RDFa Serialiserungsmöglichkeit für das Datenmodell in Abbildung~\ref{fig:rdf-intro} wurde in Listing~\ref{lst:rdfa} veranschaulicht.
\begin{listing}[H]
\begin{minted}[linenos,
numbersep=5pt,
frame=single,
framesep=2mm]{html}
<html>
  <body>
    <p>test</p>
  </body>
</html>
\end{minted}
\caption{Datenmodell in RDFa}
\label{lst:rdfa}
\end{listing}
JSON-LD kann auch benutzt werden um Webseite semantisch zu annotieren \parencite[vgl.]{Vincent2015}, aber es wurde ursprünglich entwicklet mit der Absicht Interoperable Web-Services für Linked Data (vgl. Definition in Abschnitt~\ref{sec:linked-data}) auszubauen und Linked Data in JSON-basierte Datenbanksysteme abzuspeichern \parencite[vgl.][Abs.~1]{Lanthaler:14:J}. Angenommen Ein Client schickt eine \texttt{HTTP GET} Anfrage an eine API, die JSON-LD unterstützen würde, würde die Antwort (in Bezug wieder auf das Datenmodel von Abbildung \ref{fig:rdf-intro}) Listing \ref{lst:rdfa} entsprechen.
\subsubsection{Semantik}
\label{sec:semantik}